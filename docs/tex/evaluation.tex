\section{Evalutation}

This final sections acts as an evaluation of the success of the Taxicoin project, both as a protocol and an implementation.

\subsection{Completeness of Requirements}

\enquote{How do you know that your set of requirements is complete and correct, with respect to the objective of the project?}

\subsection{Impact}

The premise of this entire project was that traditional taxi travel can be an unfair experience, and that riders and drivers are often at the mercy of large corporations, especially with more modern app-based services.

As far as taking the power away from a central authority goes, Taxicoin has achieved this by effectively utilising an Ethereum network based protocol, which doesn't place the trust of the service in any one participant's hands. In theory it should be nonsensical for one party to attempt to cheat the other, as they will lose their deposit, although the deposit must be significant enough for this mechanic to function correctly.

If a person wished to develop a Taxi app, their desired features should conceivably be possible to implement within the scope of a Taxicoin implementation. And in the case that this is not true, as the protocol specification is open source, they may submit proposed changes to it, or write an extension to it.

\subsubsection{Negotiation Standard}

Taxicoin uses its own novel method of negotiation, where two parties first agree on a price for a service off-chain (through Whisper), and then make a formal agreement on chain. This is a method which may potentially be used with a myriad of other applications.

As such, it would be sensible for a generic standard to be developed for this purpose. This would allow for the development of other contracts which can participate in negotiations which follow the standard, in a similar way to, for example, interfaces in programming languages which support polymorphism.

\subsubsection{User Value}

In context of aims set out at the start of the report.

\subsection{Future Development}

There is certainly room for improvement in both the protocol, and the implementation presented in this document.

\subsubsection{Additional Features}

There are several mentions throughout this report of potential improvements to the protocol. One such is the inclusion of a message type for informal \enquote{chat} between rider and driver before and during a journey. Although the protocol functions without this, it would be useful to be able to communicate special requirements, for example.

The Taxicoin JavaScript library makes heavy use of the \textit{Web3.js} library and signing transactions on a local Ethereum node, which functions adequately, but can be inflexible. If the user does not have access to their own local node, they may wish to use a third party node, such as the Infura service [cite]. In order for this to be secure, a user's signing key must not be transmitted to the third party. Therefore, a reasonable improvement to allow this would be to implement the signing of transactions locally (from JavaScript). This can be achieved either with Web3, or via an alternative such as \textit{Ethers.js}.

Additionally, due to the interaction-based nature of the protocol, there are likely to be some \enquote{rough edges} around the protocol, for which certain aspects will have to be adjusted. These are most likely to be focused around unforeseen exceptional circumstances.

As the protocol has been designed around an open ecosystem of independent implementations, it is hoped that such implementations will begin to be written. The core of these will likely be interaction libraries for various languages (other than JavaScript), and indeed one such library is already under development for the Go language [cite], based on the protocol specification in this document.

\subsubsection{Licensing Considerations}

At present, taxi drivers are required to hold a license in order to operate in many areas. In it's current state, the Taxicoin protocol could potentially support checking whether a driver is allowed to operate. However, this would require a blockchain-based \textit{oracle}, maintained by the body which issues taxi licenses. This would allow other smart contracts on the network to view the records of drivers (the issuing body would also have to publish an identifying address of the driver).

From a technical point of view, this is a very simple step to take. However, it would likely be blocked by local government bureaucracy, therefore it is not feasible at current, and it was not included in the project for this reason.

Identity on blockchain-based networks is still a relatively new topic. Many research projects are targeting the issue, therefore it is hoped that an agreed-upon solution and standard will be available in the near future.

\subsubsection{Client Algorithms}

The specification states that certain decisions are intended to be made in a semi-autonomous way. This includes the fare negotiation stage when a rider proposes a job to a set of drivers.

In the naive example client presented previously, the rider and driver fare negotiation is a manual process, however in reality it should function somewhat like the following: the driver's client determines an initial fare based on distance of journey and any other parameters. The rider's client then either approves or rejects, based on its threshold of what it believes a fair price to be. In the case of a rejection, the driver's client would incrementally decrease the fare quote until a threshold set by the driver is met, at which point the driver's client would reject the job.

There are many factors to be considered when developing this algorithm, thus why it was out of the scope of this project.
