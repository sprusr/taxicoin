\section{Introduction}

Taxicoin is an attempt at designing and building a protocol for hailing taxis, where the entire system is fully decentralised, with no single authority in control. The motivation behind this is to combat some of the issues found with existing similar traditional applications, such as \textit{Uber} and \textit{Lyft}.

These companies saw a way to improve the taxi industry, and by improving the user experience and ease of ordering a taxi, attracted many users. However, as was the case with existing taxi companies, they still take a significant cut of fares. Coupled with the fact that they attempt to keep fares lower for passengers, the drivers are left with very little earnings.

In contrast, Taxicoin has been designed so that the will of riders and drivers self-regulates fares. A driver decides and quotes fares on an individual basis, rather than being instructed as to what fare they should charge. Riders then only accept fares that they feel are fair -- if they are in a hurry, they may be willing to pay a higher price.

\subsection{Benefits of Decentralisation}

In general, decentralised systems are more open. In context, this means that anybody is free to participate -- one of the key precepts of Taxicoin.

Additionally, such systems, if designed well, should be not in the control of any one group of people. If we look at traditional taxi companies, the customers are at the will of the company. In many cases, one company will form the sole taxi coverage of a territory, meaning that they can decide on the price for a journey.

With a decentralised system, decisions about such issues are made transparently between all parties involved. In short, decentralisation creates a fairer system.

\subsection{Why an Open Protocol}

Continuing with the themes of decentralisation, it was important that Taxicoin be as open as possible. If a single entity was tasked with developing and maintaining the Taxicoin network, and decided to cease doing so, the entire system would likely collapse. It also creates a \textit{walled garden} scenario, where the single entity has total control over the system. They could decide at any point to disregard the original decisions made behind the network, and start charging a fee for each journey.

But with an open protocol, the specification needed to implement ones own version of Taxicoin is freely available. If users in a certain location beyond that which is actively supported by the initial version wish to launch their own Taxicoin, they can. It also allows for continued critique and improvement, where potential bugs can be discovered and fixed, and if a developer external to the original project wishes to write an extension to the protocol to add further features, they may do so.
